%
\documentclass{article}%
\usepackage{amsmath}
\usepackage{amsfonts}
\usepackage{amssymb}
\usepackage{graphicx}%
\setcounter{MaxMatrixCols}{30}
%TCIDATA{OutputFilter=latex2.dll}
%TCIDATA{Version=5.50.0.2960}
%TCIDATA{CSTFile=Literate Programming Article.cst}
%TCIDATA{Created=Monday, March 27, 2006 10:55:29}
%TCIDATA{LastRevised=Tuesday, July 07, 2015 17:31:58}
%TCIDATA{<META NAME="GraphicsSave" CONTENT="32">}
%TCIDATA{<META NAME="SaveForMode" CONTENT="3">}
%TCIDATA{BibliographyScheme=Manual}
%TCIDATA{<META NAME="DocumentShell" CONTENT="Articles\Literate Programming Article">}
%TCIDATA{Language=American English}
%BeginMSIPreambleData
\providecommand{\U}[1]{\protect\rule{.1in}{.1in}}
%EndMSIPreambleData
\newtheorem{theorem}{Theorem}
\newtheorem{acknowledgement}[theorem]{Acknowledgement}
\newtheorem{algorithm}[theorem]{Algorithm}
\newtheorem{axiom}[theorem]{Axiom}
\newtheorem{case}[theorem]{Case}
\newtheorem{claim}[theorem]{Claim}
\newtheorem{conclusion}[theorem]{Conclusion}
\newtheorem{condition}[theorem]{Condition}
\newtheorem{conjecture}[theorem]{Conjecture}
\newtheorem{corollary}[theorem]{Corollary}
\newtheorem{criterion}[theorem]{Criterion}
\newtheorem{definition}[theorem]{Definition}
\newtheorem{example}[theorem]{Example}
\newtheorem{exercise}[theorem]{Exercise}
\newtheorem{lemma}[theorem]{Lemma}
\newtheorem{notation}[theorem]{Notation}
\newtheorem{problem}[theorem]{Problem}
\newtheorem{proposition}[theorem]{Proposition}
\newtheorem{remark}[theorem]{Remark}
\newtheorem{solution}[theorem]{Solution}
\newtheorem{summary}[theorem]{Summary}
\newenvironment{proof}[1][Proof]{\textbf{#1.} }{\ \rule{0.5em}{0.5em}}
%{
\makeatletter
\typeout{litprog.sty Copyright T.S.Norvell 1997-1998.}

%%%%%%%%%%%%%%%%%%%%%%%%
% The code environment %
%%%%%%%%%%%%%%%%%%%%%%%%
\def\code{%
\bgroup%
\parsep 0pt%
\parskip 0pt%
\itemsep 0pt%
\topsep 0pt%
\parindent 0pt%
\def\descriptionlabel##1{\hspace\labelsep \bf ##1}%
\def\description{\list{}{\parsep 0pt \parskip 0pt \itemsep 0pt \topsep 0pt%
       \parindent 0pt \labelwidth\z@ \itemindent\z@ \leftmargin 2em%
       \let\makelabel\descriptionlabel}}%
\let\enddescription\endlist%
	\par\noindent\hrulefill\par\sf}%
\def\endcode{\par\noindent\hrulefill\par\egroup}

\def\hyperref#1#2#3#4{{\em #1 (\S\ref{#4})}}

\def\lpfile#1\lpendfile{{\bf File: \tt #1}}
\def\lpexpand#1\lpendexpand{{\bf $\langle$\rm(\S#1)$\rangle$}}
\def\lpaddsto#1\lpendaddsto{{\bf Adds to: \rm (\S#1)}}
\def\lpdefines#1\lpenddefines{{\bf defines: \sf #1}}
\def\lpreplace#1\lpwith#2\lpendreplace{{\bf Replace:} #1 {\bf with} #2}

\def\lptab{$\rightarrow $\quad\quad}

%%%%%%%%%%%%%%%%%%%%%%%%%%
% The indent environment %
%%%%%%%%%%%%%%%%%%%%%%%%%%

\def\indent{\description}
\def\endindent{\enddescription}

%%%%%%%%%%%%%%%%%%%%%%%
% Framed environement %
%%%%%%%%%%%%%%%%%%%%%%%
\newlength{\frameWidth}
\newlength{\frameTopExtra}
\newlength{\frameBottomExtra}
\newsavebox{\chuck}

\newenvironment{framed}
{\setlength{\frameWidth}{\linewidth}%
 \addtolength{\frameWidth}{-2\fboxsep}%
 \addtolength{\frameWidth}{-2\fboxrule}%
 \setlength{\frameTopExtra}{1.0ex}%
 \setlength{\frameBottomExtra}{0ex}%
 \setlength{\frameTopExtra}{\baselinestretch \frameTopExtra}%
 \global\setbox\chuck\vbox\bgroup\hsize\frameWidth \@parboxrestore}%
{\par\vskip\z@\egroup\shortstack{{}\\[\frameTopExtra]\fbox{\usebox\chuck}%
\\[\frameBottomExtra]{}}}

\makeatother
%}

%BeginMSIPreambleData
\ifx\pdfoutput\relax\let\pdfoutput=\undefined\fi
\newcount\msipdfoutput
\ifx\pdfoutput\undefined\else
\ifcase\pdfoutput\else
\msipdfoutput=1
\ifx\paperwidth\undefined\else
\ifdim\paperheight=0pt\relax\else\pdfpageheight\paperheight\fi
\ifdim\paperwidth=0pt\relax\else\pdfpagewidth\paperwidth\fi
\fi\fi\fi
%EndMSIPreambleData
\begin{document}

\title{Language Design for CGRA\ project. Design 10 [Draft 20].}
\author{Theodore S\ Norvell\\Electrical and Computer Engineering\\Memorial University}
\maketitle

\begin{abstract}
Abstract to be done.

\end{abstract}

Change History

\begin{itemize}
\item 2006 Sept 18 Version 3

\item 2007 August. Version 4.

\item 2007 November. Version 5. Added constants

\item Changed syntax of elseif

\item Allowed name at the end of a class or interface

\item 2008 Nov 3. Version 6

\begin{itemize}
\item Replaced atomic command with \textquotedblleft with
command\textquotedblright

\item Added section on parallel access to locations
\end{itemize}

\item 2011. July. Added information on expressions

\item 2013 Summer. Changes related to implementation. Including

\begin{itemize}
\item More detail on expressions

\item Minor changes to syntax

\item Moved constructor parameters
\end{itemize}

\item 2015

\begin{itemize}
\item More information on expressions
\end{itemize}
\end{itemize}

Meta notation%
\[%
\begin{tabular}
[c]{ll}%
$N\rightarrow E$ & An $N$ can be $E$\\
$\underline{(}E\underline{)}$ & Grouping\\
$E\;F$ & An $E$ followed by an $F$\\
$E^{\ast}$ & Zero or more $E$s\\
$E^{\ast F}$ & Zero or more $E$s separated by $F$s\\
$E^{+}$ & One or more $E$s\\
$E^{+F}$ & One or more $E$s separated by $F$s\\
$E^{?}$ & Zero or one $E$s\\
$\underline{\lbrack}E\underline{]}$ & Zero or one $E$s\\
$E\mid F$ & Either an $E$ or a $F$%
\end{tabular}
\
\]


\section{Classes and Objects}

\subsection{Programs}

A program is a sequence of classes, interfaces, and objects.%
\[
\mathit{Program}\rightarrow\underline{(}\mathit{ClassDecl}\mid
\mathit{IntfDecl}\mid\mathit{ObjectDecl}\mid\mathbf{;}\underline{)}^{\ast
}\;\mathbf{eof}%
\]


If multiple files are compiled together the sequence is simply the
concatenation of the sequences in the individual files.

\subsection{Types}

Types come in several categories.

\begin{itemize}
\item Primitive types:\ Primitive types represent sets of value. As such they
have no mutators. However objects of primitive types may be assigned to to
change their values. Primitive types represent such things as numbers. They include

\begin{itemize}
\item Int8, Int16, Int32, Int64

\item Real16, Real32, Real64

\item Bool
\end{itemize}
\end{itemize}

The names of the types specify the minimum number of bits. However they do not
specify the exact number of bits. E.g. int16 might be represented by 32 bit
locations.\ The names of the primitive types are not reserved words.

\begin{itemize}
\item Classes:\ Classes represent sets of objects. As such they support
methods that may change the object's state.

\item Interfaces. Interfaces are like classes, but without the implementation.

\item Arrays:\ Arrays may be arrays of primitives or arrays of objects.

\item Generic types. Generic types are not really types at all, but rather
functions from types to types. In order to be used, generic types must be instantiated.
\end{itemize}

Types are either names of classes, names of interfaces, array types or
specializations of generic types%
\[
\mathit{Type}\rightarrow\mathit{Name}\mid\mathit{Type}\;\mathit{GArgs}%
\mid\mathit{Type\ }\mathbf{[\ }\mathit{Bounds}\ \mathbf{]}%
\]


Arrays are 1 dimensional and indexed from $0$, so the bounds are simply one
number%
\[
\mathit{Bounds}\rightarrow\mathit{ConstExp}%
\]


\subsubsection{Restrictions}

The bounds of an array must be an integral expression greater than or equal to zero.

For a type of the form $\mathit{Type}\;\mathit{GArgs},$the name must refer to
a generic type and the generic arguments must be the same in number as the
type's parameters. Each argument must not be generic.

TODO More to say?

\subsection{Objects}

Objects are named instances of types.%
\[
\mathit{ObjectDecl}\rightarrow(\mathbf{const}\mid\mathbf{obj})\;\mathit{Name}%
\;\underline{[}:\mathit{Type}\underline{]}:=\mathit{InitExp}%
\]


\subsubsection{Restrictions}

If the type is missing it is considered to be the same as that of the
$\mathit{InitExp}$. If the type is not missing, it must be a supertype of the
type of the $\mathit{InitExp}$. It follows that the type may not be generic.

If the type (or inferred type) is primitive, the declaration creates a named
location. If the type is an array type, the declaration names an array of
locations. If the type is a class, the declaration names an object.

If the declaration is \textbf{const}, the type must be primitive and the named
location is a constant location that may not be changed.

\subsection{Initialization expressions}

\subsubsection{Constant initialization expression}%

\[
\mathit{InitExp}\rightarrow\mathit{ConstExp}%
\]
The type of the $\mathit{InitExp}$ is the same as the $\mathit{ConstExp}$ in
the first case.

\subsubsection{Array Initialization Expressions}%

\[
\mathit{InitExp}\rightarrow\left(  \mathbf{for}\;\mathit{Name}:\mathit{Bounds}%
\;\mathbf{do}\;\mathit{InitExp}\;\underline{[}\mathbf{for}\underline{]}%
\right)
\]
The type of an $\mathit{ArrayInit}$ is array $k$ of $t$ where $k$ is the value
of the bounds and $t$ is the value of the $\mathit{InitExp}$.

\subsubsection{New Initialization Expressions}%

\[
\mathit{InitExp}\rightarrow\mathbf{new}\;\mathit{Type}(\mathit{ConstExp}%
^{\ast,})
\]
In this case the $\mathit{Type}$ must be a nongeneric class type, the number
of arguments must match the number of parameters for the class's constructor.
For each parameter (whether \textbf{in} or \textbf{obj}), the argument's type
must be a subtype of the parameter's type.

\subsubsection{Choice Initialization Expressions}%

\[
\mathit{InitExp}\rightarrow\left(  \mathbf{if}~\mathit{ConstExp}%
\;\mathbf{then}\;\mathit{InitExp}\;\underline{(}\mathbf{else\;if}%
\;\mathit{ConstExp}\;\mathbf{then}\;\mathit{InitExp}\underline{)}^{\ast
}\;\mathbf{else}\;\mathit{InitExp}\;\underline{[}\mathbf{if}\underline{]}%
\right)
\]


The type of the expression is the least supertype of all the \textit{InitExp}s.

\paragraph{Restrictions}

The type of each $\mathit{ConstExp}$ must be Bool.

If no such supertype exists, then there is an error.

\subsection{Classes and interfaces}

Each class declaration defines a family of types. Classes may be generic or
nongeneric. A generic class has one or more generic parameters%

\begin{align*}
ClassDecl  &  \rightarrow(\mathbf{class}\;Name\;GParams^{?}\;(CParam^{\ast
,})\;\underline{(}\mathbf{implements}\;Type^{+,}\underline{)}^{?}\mathbf{\;}\\
&  \underline{(}ClassMember\underline{)}^{\ast}\;\underline{[}\mathbf{class~}%
\underline{[}Name\underline{]}\underline{])}%
\end{align*}


\begin{itemize}
\item The $Name$ is the name of the class.

\item The $GParams$ is only present for generic classes, which will be
presented in a later section.

\item The $Type$s are the interfaces that the class implements.
\end{itemize}

An interface defines a type. Interfaces may be generic or nongeneric. A
generic interfaces has one or more generic parameters%

\[
IntfDecl\rightarrow\left(  \mathbf{interface}\;Name\;GParams^{?}%
\;\underline{(}\mathbf{extends}\;Type^{+,}\underline{)}^{?}\;\underline{(}%
IntfMember\underline{)}^{\ast}\;\underline{[}\mathbf{interface~}%
\underline{[}Name\underline{]}\underline{]}\right)
\]


\begin{itemize}
\item The $Name$ is the name of the class.

\item The $GParams$ will be presented in a later section.

\item The $Type$s are the interfaces that the interface extends.
\end{itemize}

Constructor parameters represent objects to which this object is connected.%
\[
CParam\rightarrow\mathbf{obj}\;Name:Type\mid\mathbf{in}\;Name:Type
\]


\begin{itemize}
\item $_{{}}$Object parameters represent named connections to other objects.
So for example if we have

\begin{code}
(class B(\ obj x :\ A\ )\ ... )

obj a := (for i\ :\ 10 do new A() )

obj b := (for i :\ 10 do new B(a(i) )
\end{code}

Then object \textsf{b[0]} knows object \textsf{a[0]} by the name of \textsf{x}.

\item In parameters are compile time constants and the corresponding argument
must be such.
\end{itemize}

\subsection{Class Members}

Class members can be fields, methods, and threads. [Nested classes and
interfaces are a possibility for the future.]%
\[
ClassMember\rightarrow Field\mid Method\mid Thread\mid{};
\]


Fields are objects, arrays, or locations that are within objects. Field
declarations therefore define the part/whole hierarchy.
\begin{align*}
Field  &  \rightarrow Access^{?}\;ObjectDecl\\
Access  &  \rightarrow\mathbf{private}\mid\mathbf{public}%
\end{align*}


Method declarations declare the method, but not its implementation. The
implementation of each must be embedded within a thread.%
\begin{align*}
Method  &  \rightarrow Access\;\mathbf{proc}\;Name(\underline{(}%
Direction\;Name:\;Type\underline{)}^{\ast,})\underline{]}\\
Direction  &  \rightarrow\mathbf{in}\mid\mathbf{out}%
\end{align*}
The types of method parameters must be primitive.

Recommended order of declarations is

\begin{itemize}
\item public methods and fields, followed by

\item private methods and fields, followed by

\item threads.
\end{itemize}

There is no `declaration before use rule'. Name lookup works from inside out.

\subsection{Interface Members}

Interface's members can be fields and methods. [Nested classes and interfaces
are a possibility for the future.]%
\[
IntfMember\rightarrow Field\mid Method\mid{};
\]


\section{Threads}

Threads are blocks executed in response to object creation.
\[
Thread\rightarrow(\mathbf{thread}\;Block\;\underline{[}\mathbf{thread}%
\underline{]})
\]


Each object contains within it zero or more threads. Coordination between the
threads within the same object are the responsibility of the programmer. All
concurrency within an object arises from the existence of multiple threads in
its class. Thus you can write a monitor (essentially) by having only one
thread in a class.

\subsection{Statements and Blocks}

A block is simply a sequence of statements and semicolons%
\[
Block\rightarrow Command\;Block\mid LocalDeclaration\mid;\;Block\mid
\]


Statements as follow

\begin{itemize}
\item Assignment statements%
\begin{align*}
Command  &  \rightarrow(ObjectIds):=(Expressions)\\
Command  &  \rightarrow ObjectId:=Expression\\
ObjectIds  &  \rightarrow\underline{(}ObjectId\underline{)}^{+,}\\
Expressions  &  \rightarrow\underline{(}Expression\underline{)}^{+,}\\
ObjectId  &  \rightarrow Name\mid ObjectId[Expression]\mid ObjectId.Name
\end{align*}
The type of the ObjectId must admit assignment, which means it should be a
primitive type, like \textbf{int32} or \textbf{real64}.

\item Local variable declaration%
\[
LocalDeclaration\rightarrow(\mathbf{obj}\mid\mathbf{const})\;Name\underline{[}%
:Type\underline{]}:=Exp\;Block
\]
Note that the initialization expression is a regular expression, not an
initExp. The initialization is performed each time the declaration is
encountered. The type may be omitted, in which case it is inferred from the
initialization expression. In either case, the type must be primitive. The
scope of a local variable name is the block that follows it. Local variables
marked \textbf{const} may not be assigned to. Names must be unique within the thread.

\item Method call statements%
\begin{align*}
Command  &  \rightarrow ObjectId.Name(Args)\\
&  \mid Name(Args)\\
Args  &  \rightarrow\underline{[}Expressions\underline{]}%
\end{align*}


Restrictions:

\begin{itemize}
\item For the first syntax, the type of the $ObjectId$ must be of class or
interface type and the $Name$ must name a public method of that type
(including inherited methods).

\item For the second syntax, the $Name$ must be the name of a (public or
private) method of the class or interface type currently being defined.

\item The number of argument must equal the number of parameters declared for
the method.

\item For each \textbf{in} parameter of the method, the corresponding argument
must have either the same type as the parameter's type or a type that can be
widened to the parameter's type.

\item For each \textbf{out} parameter of the method, the argument must refer
to a location and the type of that location must be the same as the type of
the argument, or type that can be widened to the argument type.
\end{itemize}

\item Sequential control flow%
\begin{align*}
Command &  \rightarrow\left(  \mathbf{if}~Expression\;\mathbf{then}%
\;Block\;\underline{(}\mathbf{else\;if}~Expression\;\mathbf{then}%
\;Block\underline{)}^{\ast}\underline{(}\mathbf{else}\;Block\underline{)}%
^{?}~\underline{[}\mathbf{if}\underline{]}\right)  \\
&  \mid\left(  \mathbf{while}~Expression\;\mathbf{do}\;Block\;\underline{[}%
\mathbf{while}\underline{]}\right)  \\
&  \mid\left(  \mathbf{for}\;Name:Bounds\;\mathbf{do}\;Block\;\underline{[}%
\mathbf{for}\underline{]}\right)
\end{align*}


\item Parallelism (note that the 2 vertical bars here is a terminal)%
\begin{align*}
Command &  \rightarrow\left(  \mathbf{co}\;Block\;\underline{(}%
||\;Block\underline{)}^{+}\;\underline{[}\mathbf{co}\underline{]}\right)  \\
&  \mid\left(  \mathbf{co}\;Name:Bounds\;\mathbf{do}\;Block\;\underline{[}%
\mathbf{co}\underline{]}\right)
\end{align*}
In the second case, the \textit{Bounds} must be compile-time constant.

\item Method implementation (note that the vertical bar here is a terminal)%
\begin{align*}
Command  &  \rightarrow\left(  \mathbf{accept}\;MethodImp\;\underline{(}%
|\;MethodImp\underline{)}^{\ast}\;\underline{[}\mathbf{accept}\underline{]}%
\right) \\
MethodImp  &  \rightarrow Name(\;\underline{(}%
Direction\;Name:Type\underline{)}^{\ast,}\;)\;\underline{[}Guard\underline{]}%
\;Block_{0}\;\underline{[}\mathbf{then\;}Block_{1}\underline{]}\\
Guard  &  \rightarrow\mathbf{when}\;Expression
\end{align*}


\begin{itemize}
\item Restrictions

\begin{itemize}
\item The names, directions and types must match the declaration.

\item The guard expression must be boolean.

\item Each method may only be implemented once per class
\end{itemize}

\item Possible restrictions:

\begin{itemize}
\item The guard may not refer to any parameters.

\item The guard may refer only to the in parameters.
\end{itemize}

\item Semantics:\ A thread that reaches an accept command must wait until
there is a call to one of the methods it implements and the corresponding
guard is true. Once there is at least one method the accept can execute, one
is selected. Input parameters are passed in, $Block_{0}$ is executed and
finally the output parameters are copied back to the calling thread. If there
is a $Block_{1}$ it is executed next.
\end{itemize}

\item Locking%
\[
Command\rightarrow\left(  \mathbf{with}\;Exp\;\underline{[}Guard\underline{]}%
\;\mathbf{do}\;Block\;\underline{[}\mathbf{with}\underline{]}\right)
\]


\begin{itemize}
\item Restrictions:

\begin{itemize}
\item The \textit{Exp} must refer to an object of type Lock.

\item The guard expression must be boolean
\end{itemize}

\item Semantics:\ 

\begin{itemize}
\item The block (including the guard) is executed as if atomically with
respect to other with commands sharing the same lock

\item If there is no guard, it defaults to true.

\item If the guard is false, then the lock is unlocked and then everything
starts again.

\item In summary the sematics is like this

\begin{code}
lock(\textit{Exp})

(\textbf{while} \textbf{not} \textit{Guard} \textbf{do} unlock(\textit{Exp})
lock(\textit{Exp})\ )

\textit{Block}

unlock(\textit{Exp})
\end{code}
\end{itemize}
\end{itemize}
\end{itemize}

\section{Parallel access to data locations}

Each object of a primitive type, including array items, is a separate
location. Access to a location is either a read access or a write access.
Accesses are not considered to be atomic, but rather to take a span of time.
As such two accesses that could overlap in time must not be to the same
location --- except that we will allow read accesses to overlap. If two
accesses could overlap in time, we say that they \textquotedblleft could
happen at the same time\textquotedblright. Suppose that $a$ and $b$ are two
accesses from separate threads that could happen at the same time

\begin{itemize}
\item Parallel read accesses are allowed. If $a$ and $b$ are both reads,
behaviour is well defined.

\item Parallel write accesses are not allowed. If $a$ and $b$ are both writes,
then behaviour is undefined.

\item Parallel read and write accesses are not allowed. If $a$ is a read and
$b$ is a write (or the other way around), then behaviour is undefined.
\end{itemize}

The compiler may or may not diagnose undefined behaviour. The reason is that
aliasing makes it impossible to tell for sure whether the behaviour of a
program is well defined or undefined. Consider

\begin{code}
(\textbf{co} $a[i]:=0\;||\;a[j]:=0$ \textbf{co})
\end{code}

\noindent In states where $i=j$, behavour is undefined. In states where
$\not i  \neq j$, but where $i$ and $j$ are in bounds, behaviour is well
defined. Since the values of $i$ and $j$ are (in general) unknowable at
compile time, the compiler is not in a position to diagnose undefined
behaviour, although a good compiler may warn that it can not rule out
undefined behaviour. Note that the fact that both statements are writing the
same value makes no difference to whether the parallel accesses are well-defined.

The programmer may prevent accesses from occuring at the same time using locks:

\begin{code}
(\textbf{co} (\textbf{with} $l$ $a[i]:=0$ )$\;||\;$(\textbf{with} $l$
$a[i]:=1$ ) )
\end{code}

\noindent In this example, the parallel accesses are well defined because they
can not take place at the same time.

Accesses may also be protected via other synchronization mechanisms. For example

\begin{code}
(\textbf{co} $s.p()\;a[i]:=0\;s.v()\;||\;s.p()\;a[i]:=1\;s.v()$ \textbf{co})
\end{code}

\noindent Is well defined if the $p$ and $v$ methods of object $s$ somehow
prevent the accesses from occurring at the same time. Method calls serve as
\textquotedblleft synchonization points\textquotedblright.

Here is one more example.

\begin{code}
\textbf{obj} $l:\mathit{Lock}:=\mathbf{new}$ $\mathit{Lock()}$

\textbf{obj} $s:=1$

(\textbf{co}

\begin{indent}
\item (\textbf{with} $l$ \textbf{when} $s=1\;s:=0$\ )

\item $a[i]:=0$

\item (\textbf{with} $l$ $s:=1$)
\end{indent}

%

%TCIMACRO{\TEXTsymbol{\vert}}%
%BeginExpansion
$\vert$%
%EndExpansion%
%TCIMACRO{\TEXTsymbol{\vert}}%
%BeginExpansion
$\vert$%
%EndExpansion


\begin{indent}
\item (\textbf{with} $l$ \textbf{when} $s=1\;s:=0$\ )

\item $a[i]:=1$

\item (\textbf{with} $l$ $s:=1$)
\end{indent}

\textbf{co})
\end{code}

\noindent In this case, the assignments to $a[i]$ can not happen at the same
time and so the result is well defined (though nondeterministic). In
particular the compiler can not move the assignment to $a[i]$ any earlier or
later in the threads. This means that with statements, like method calls,
serve as \textquotedblleft synchronization points.\textquotedblright

One might at first think that there is no need to protect the assignments
$s:=1$, in the previous example, with \textbf{with} statements. This is not
so; there is a read access in the other process that could happen at the same
time. The following program's behaviour is undefined.

\begin{code}
\textbf{obj} $l:\mathit{Lock}$

\textbf{obj} $s:=1$

(\textbf{co}

\begin{indent}
\item (\textbf{with} $l$ \textbf{when} $s=1\;s:=0$\ )

\item $a[i]:=0$

\item $s:=1$ // Wrong. Access is unprotected.
\end{indent}

%

%TCIMACRO{\TEXTsymbol{\vert}}%
%BeginExpansion
$\vert$%
%EndExpansion%
%TCIMACRO{\TEXTsymbol{\vert}}%
%BeginExpansion
$\vert$%
%EndExpansion


\begin{indent}
\item (\textbf{with} $l$ \textbf{when} $s=1\;s:=0$\ )

\item $a[i]:=1$

\item $s:=1$ // Wrong! Access is unprotected.
\end{indent}

\textbf{co})
\end{code}

\noindent A compiler optimizing this program, might look at the sequence
$a[i]:=1\,s:=1$ and decide to reorder the statements to $s:=1\,a[i]:=1$ and
that would be legitimate in the sense that any change to an undefined program
can not make it more wrong.

While sequential programming constructs impose an nominal order on execution,
a compiler is welcome to reorder accesses that are to different locations (or
that are both read accesses). For example the program $a:=1\,b:=1$ says that
$a$ should be written first, nominally. However as $a$ and $b$ are different
locations, the command can be rewritten to $b:=1\,a:=1$. The compiler can
assume that there are no parallel accesses to $a$ or $b$ that could happen at
the same time, as such accesses would make the program undefined and there is
nothing the compiler can do to an undefined program to make it more wrong.
Thus the compiler is welcome to make any sequential optimizations to code that
appears between synchronization points. The following are synchronization points:

\begin{itemize}
\item The start of a with command.

\item The end of a with command

\item The start of an accept command

\item The return from an accept command

\item Any method call (after argument evaluation)
\end{itemize}

\section{Expressions}

\subsection{Syntax}

Expressions%
\begin{align*}
Exp  &  \rightarrow Exp0~\underline{\underline{(}}\underline{(}\text{=%
%TCIMACRO{\TEXTsymbol{>}}%
%BeginExpansion
$>$%
%EndExpansion
}\mid\text{%
%TCIMACRO{\TEXTsymbol{<}}%
%BeginExpansion
$<$%
%EndExpansion
=}\mid\text{%
%TCIMACRO{\TEXTsymbol{<}}%
%BeginExpansion
$<$%
%EndExpansion
=%
%TCIMACRO{\TEXTsymbol{>}}%
%BeginExpansion
$>$%
%EndExpansion
}\underline{)}~Exp0\underline{)}^{\ast}\;\underline{[}\mathbf{as}%
\mathit{\;Id}\underline{]}\\
Exp0  &  \rightarrow Exp1~\underline{\underline{(}}\underline{(}\text{%
%TCIMACRO{\TEXTsymbol{\backslash}}%
%BeginExpansion
$\backslash$%
%EndExpansion
/}\mid\text{or}\underline{)}~Exp1\underline{)}^{\ast}\\
Exp1  &  \rightarrow Exp2~\underline{\underline{(}}\underline{(}\text{/%
%TCIMACRO{\TEXTsymbol{\backslash}}%
%BeginExpansion
$\backslash$%
%EndExpansion
}\mid\text{and}\underline{)}~Exp2\underline{)}^{\ast}\\
Exp2  &  \rightarrow Exp3\mid\mathrm{not}\;Exp2\mid\mathrm{\symbol{126}%
}\;Exp2\\
Exp3  &  \rightarrow Exp4~\underline{\underline{(}}\underline{(}\text{=}%
\mid\text{\symbol{126}=}\mid\text{not=}\mid\text{%
%TCIMACRO{\TEXTsymbol{<}}%
%BeginExpansion
$<$%
%EndExpansion
}\mid\text{\_%
%TCIMACRO{\TEXTsymbol{<}}%
%BeginExpansion
$<$%
%EndExpansion
}\mid\text{%
%TCIMACRO{\TEXTsymbol{>}}%
%BeginExpansion
$>$%
%EndExpansion
\_}\mid\text{%
%TCIMACRO{\TEXTsymbol{>}}%
%BeginExpansion
$>$%
%EndExpansion
}\underline{)}~Exp4\underline{)}^{\ast}\\
Exp4  &  \rightarrow Exp5~\underline{(}\underline{(}+\mid-\underline{)}%
~Exp5\underline{)}\ast\\
Exp5  &  \rightarrow Primary~\underline{(}\underline{(}\ast\mid/\mid
\operatorname{div}\mid\operatorname{mod}\underline{)}~Primary\underline{)}%
^{\ast}\\
Exp6  &  \rightarrow -Exp6\mid Exp7\\
Exp7  &  \rightarrow Primary\\
Primary  &  \rightarrow(Exp) \mid ObjectId \mid Literal
\end{align*}


Operator precedence is as above. Binary operators%
\[
\text{\textrm{=%
%TCIMACRO{\TEXTsymbol{>} }%
%BeginExpansion
$>$
%EndExpansion
%TCIMACRO{\TEXTsymbol{<}}%
%BeginExpansion
$<$%
%EndExpansion
=
%TCIMACRO{\TEXTsymbol{<}}%
%BeginExpansion
$<$%
%EndExpansion
=%
%TCIMACRO{\TEXTsymbol{>} }%
%BeginExpansion
$>$
%EndExpansion
%TCIMACRO{\TEXTsymbol{\backslash}}%
%BeginExpansion
$\backslash$%
%EndExpansion
/ or /%
%TCIMACRO{\TEXTsymbol{\backslash} }%
%BeginExpansion
$\backslash$
%EndExpansion
and + - * / div mod}}%
\]
are left associative. The following are synonymous (see also
%TCIMACRO{\hyperref{Lexical Issues}{section}{}{sec:LexicalIssues}}%
%BeginExpansion
section\ref{sec:LexicalIssues}
%EndExpansion
).%
\[%
\begin{tabular}
[c]{llll}%
=%
%TCIMACRO{\TEXTsymbol{>}}%
%BeginExpansion
$>$%
%EndExpansion
&  & $\Rightarrow$ & \\%
%TCIMACRO{\TEXTsymbol{<}}%
%BeginExpansion
$<$%
%EndExpansion
= &  & $\Leftarrow$ & \\%
%TCIMACRO{\TEXTsymbol{<}}%
%BeginExpansion
$<$%
%EndExpansion
=%
%TCIMACRO{\TEXTsymbol{>}}%
%BeginExpansion
$>$%
%EndExpansion
&  & $\Leftrightarrow$ & \\%
%TCIMACRO{\TEXTsymbol{\backslash}}%
%BeginExpansion
$\backslash$%
%EndExpansion
/ & or & $\vee$ & \\
/%
%TCIMACRO{\TEXTsymbol{\backslash}}%
%BeginExpansion
$\backslash$%
%EndExpansion
& and & $\wedge$ & \\
\symbol{126}= & not= & $\neq$ & $\lnot=$\\
\symbol{126} & not & $\lnot$ & \\
\_%
%TCIMACRO{\TEXTsymbol{<}}%
%BeginExpansion
$<$%
%EndExpansion
&  & $\leq$ & $\leqslant$\\%
%TCIMACRO{\TEXTsymbol{>}}%
%BeginExpansion
$>$%
%EndExpansion
\_ &  & $\geq$ & $\geqslant$%
\end{tabular}
\]


Binary operators%
\[
\text{\texttt{=\quad\symbol{126}=\quad%
%TCIMACRO{\TEXTsymbol{<}}%
%BeginExpansion
$<$%
%EndExpansion
\quad\_%
%TCIMACRO{\TEXTsymbol{<}}%
%BeginExpansion
$<$%
%EndExpansion
\quad%
%TCIMACRO{\TEXTsymbol{>}}%
%BeginExpansion
$>$%
%EndExpansion
\quad%
%TCIMACRO{\TEXTsymbol{>}}%
%BeginExpansion
$>$%
%EndExpansion
\_}}%
\]
are \textquotedblleft chaining\textquotedblright, meaning that any sequence of
these operators is `anded' together. For example expression%
\[
\text{a \_%
%TCIMACRO{\TEXTsymbol{<}}%
%BeginExpansion
$<$%
%EndExpansion
\ b
%TCIMACRO{\TEXTsymbol{<}}%
%BeginExpansion
$<$%
%EndExpansion
\ c = d \symbol{126}= e}%
\]
is logically equivalent to
\[
\text{a \_%
%TCIMACRO{\TEXTsymbol{<}}%
%BeginExpansion
$<$%
%EndExpansion
\ b and b
%TCIMACRO{\TEXTsymbol{<}}%
%BeginExpansion
$<$%
%EndExpansion
\ c and c = d and d \symbol{126}= e}%
\]
except that for the former, all operands are first converted to a common type.

Expressions are not necessarily evaluated from left to right or in any
particular order. Associativity of associative operators may or may not be
respected. Logical operators are not (necessarily)\ short-circuiting.

A constant expression is the same as an expression except it must be evaluable
at elaboration time. Thus a constant expression can depend on \textbf{in}
parameters of the constructor.%
\[
ConstExp\rightarrow Exp
\]


\subsection{Literals}

Decimal integer literals are written as a series of decimal digits with any
number of internal underlines. For example \textquotedblleft%
0\textquotedblright, \textquotedblleft123\textquotedblright, \textquotedblleft%
123\_456\textquotedblright.

Integer literals in bases 2, 8, and 16 are also allowed. These are written
with the base first, followed by a \# symbol, followed by the digits that make
up the literal. Again underscores can appear internally to provide spacing.
Here is the lexical syntax of integer literals%
\begin{align*}
\mathit{DecimalLiteral}  &  \rightarrow D\mid D\underline{(}D\mid
\text{\texttt{\_}}\underline{)}^{\ast}D\\
\mathit{HexLiteral}  &  \rightarrow\text{\texttt{16\#}}\underline{(}H\mid
H(H\mid\text{\texttt{\_}}\underline{)}^{\ast}H\\
\mathit{OctLiteral}  &  \rightarrow\text{\texttt{8\#}}\underline{(}O\mid
O(o\mid\text{\texttt{\_}}\underline{)}^{\ast}O\\
\mathit{BinLiteral}  &  \rightarrow\text{\texttt{2\#}}\underline{(}B\mid
B(B\mid\text{\texttt{\_}}\underline{)}^{\ast}B\\
D  &  \rightarrow0\mid1\mid\cdots\mid9\\
H  &  \rightarrow0\mid1\mid\cdots\mid9\mid a\mid\cdots\mid f\mid A\mid
\cdots\mid F\\
O  &  \rightarrow0\mid1\mid\cdots\mid7\\
B  &  \rightarrow0\mid1
\end{align*}


Floating point (or real) literals are defined by the following lexical rules%
\begin{align*}
\mathit{RealLiteral}  &  \rightarrow\text{\texttt{.}}\;\mathit{DecimalLiteral}%
\;\underline{[}\mathit{Exp}\underline{]}\\
&  \mid\mathit{DecimalLiteral}\;\mathtt{.}\;\underline{[}\mathit{Exp}%
\underline{]}\\
&  \mid\mathit{DecimalLiteral}\;\mathtt{.}\;\mathit{DecimalLiteral\;}%
\underline{[}\mathit{Exp}\underline{]}\\
&  \mid\mathit{DecimalLiteral}\;\mathit{Exp}\\
\mathit{Exp}  &  \rightarrow\underline{(}\mathtt{e}\mid\mathtt{E}%
\underline{)}\;\underline{[}\mathtt{+}\mid\mathtt{-}\underline{]}%
\;\mathit{DecimalLiteral}%
\end{align*}


\subsubsection{Restrictions}

Decimal, binary, octal, and hex, literals must be repesentable in 63 bits,
i.e. they must be less than $2^{63}.=\allowbreak9,223,372,036,854,775,808$.
Real literals must be representable in 64 bits on both the compiling and the
target machines.

\subsection{Types}

The arithmetic primitive types include

\begin{itemize}
\item Int8, Int16, Int32, Int64

\item Real16, Real32, Real64
\end{itemize}

\noindent and have the following one step widening relationships

\begin{itemize}
\item Int8 $\rightarrow$ Int16 $\rightarrow$ Int32 $\rightarrow$ Int64

\item Real16 $\rightarrow$ Real32 $\rightarrow$ Real64

\item Int8 $\rightarrow$ Real16

\item Int16 $\rightarrow$ Real32

\item Int32 $\rightarrow$ Real64
\end{itemize}

\subsubsection{Object\ IDs}

The type of an ObjectID is the discussed in [[section on objectIDs above]]

\subsubsection{Literals}

The type of any real literal is of type Real64. However, when an
initialization of an object consists soley of a real literal, and the object
is explicitly typed, the literal is typed as the type of the object, provided
that type is Real16 or Real32. Effectively there is an implicit `as'. For example

\begin{code}
\textbf{obj} a :\ Real32 := 1.0
\end{code}

\noindent is equivalent to

\begin{code}
\textbf{obj} a :\ Real32 := 1.0 \textbf{as} Real32
\end{code}

The type of any integer literal\ (i.e. decimal, binary, octal, or hex literal)
is Int32 unless its value is over $2^{31}=2,147,483,648$, in which case it is
of type Int64. However, when an initialization of an object consists soley of
an integer literal, and the object is explicitly typed, the literal is typed
as the type of the object, provided that type is Int8, Int16, or Real16 or
Real32. Effectively there is an implicit `as'. For example

\begin{code}
\textbf{obj} a :\ Real32 := 1 \textbf{obj} b :\ Int8 := 0
\end{code}

\noindent is equivalent to

\begin{code}
\textbf{obj} a :\ Real32 := 1 \textbf{as} Real32 \textbf{obj} b :\ Int8 := 0
\textbf{as} Int8
\end{code}

\subsubsection{Binary and chaining operators}

When applying binary operations, operands are generally widened to the
narrowest common widening (ncw). For example if a Real32 is compared to an
Int32, the narrowest common widening is Real64, so both operands are converted
to Real64 and are compared at that type. Note that the ncr is not always
defined. In particular $\mathrm{ncw}(\mathrm{Int64},t)$ is not defined for any
$t\in\left\{  \mathrm{Real16},\mathrm{Real32},\mathrm{Real64}\right\}  $.

An integral type is any of the types Int8, Int16, Int32, Int64.

Binary operators behave according to this table in which B stands for Boolean,
A for arithmetic, I\ for integral, ncw for narrowest common widening, and ncrw
stands for narrowest common real widening.%

\[%
\begin{tabular}
[c]{lllll}%
Operator & Left & Right & Operation Type & Result Type\\%
%TCIMACRO{\TEXTsymbol{<}}%
%BeginExpansion
$<$%
%EndExpansion
= =%
%TCIMACRO{\TEXTsymbol{>} }%
%BeginExpansion
$>$
%EndExpansion
%TCIMACRO{\TEXTsymbol{<}}%
%BeginExpansion
$<$%
%EndExpansion
=%
%TCIMACRO{\TEXTsymbol{>} }%
%BeginExpansion
$>$
%EndExpansion
/%
%TCIMACRO{\TEXTsymbol{\backslash} }%
%BeginExpansion
$\backslash$
%EndExpansion
%TCIMACRO{\TEXTsymbol{\backslash}}%
%BeginExpansion
$\backslash$%
%EndExpansion
/ & B & B & B & B\\
= \symbol{126}=
%TCIMACRO{\TEXTsymbol{>} }%
%BeginExpansion
$>$
%EndExpansion
%TCIMACRO{\TEXTsymbol{<}}%
%BeginExpansion
$<$%
%EndExpansion
\
%TCIMACRO{\TEXTsymbol{>}}%
%BeginExpansion
$>$%
%EndExpansion
\_ \_%
%TCIMACRO{\TEXTsymbol{<}}%
%BeginExpansion
$<$%
%EndExpansion
& A & A & ncw & B\\
+ - * & A & A & ncw & ncw\\
/ & A & A & ncrw & ncrw\\
div mod & I & I & ncw & ncw
\end{tabular}
\
\]
For the comparison operations, when multiple operations are chained together,
as in $a\leq b<c$, all operands are widened to the narrowest common type
before any comparisons are made. If that narrowest common widening does not
exist, then it is an error.

For division with /, if both operands are integral, then the result is the
narrowest real type that is wider than both operands.

\textbf{Restrictions}:

\begin{itemize}
\item A binary operation is not allowed if the types of the operands are not
as shown in the above table.

\item A binary operation is not allowed if the ncw or ncrw is not defined for
the operand types.
\end{itemize}

\subsubsection{Unary operators}

Unary operators do not force conversions. The result type is the same as the
operand type.%
\[%
\begin{tabular}
[c]{lll}%
Operator & Left & Result\\
\symbol{126} & B & B\\
$-$ & A & A
\end{tabular}
\]


\textbf{Restriction}:

\begin{itemize}
\item A unary operation is not allowed if the type of the operand is not as
shown in the above table.
\end{itemize}

\subsubsection{`As' expressions}

The `as' construct forces a value to be reinterpreted as another type. The
identifier must be one of the primitive types. Any expression of an arithmetic
type can be reinterpreted as a value of any other arithemetic type.

\textbf{Restrictions}:

\begin{itemize}
\item The `as' construct is not allowed if the operand is not of arithmetic type.

\item The `as'construct is not allowed if the identifier on the right is does
not resolve to one of the arithmetic types.
\end{itemize}

\subsection{Semantics}

Each integral type is associated with a particular range of values. These are

\begin{itemize}
\item Int8 $\left\{  -128,..,127\right\}  $

\item Int16 $\left\{  -32\_768,..,32\_767\right\}  $

\item Int32 $\left\{  -2\_147\_483\_648,..,2\_147\_483\_647\right\}  $

\item Int64 $\left\{
-9\_223\_372\_036\_854\_775\_808,..,9\_223\_372\_036\_854\_775\_807\right\}  $
\end{itemize}

The semantics of operations on intergral types is simply that the result is
the mathematical answer if representable in the ncw type and is not defined
otherwise. For example%
\[
\left(  1\;\text{as Int8}\right)  +\left(  127\;\text{as Int8}\right)
\]
is not defined since the ncw type is Int8 and 128 lies outside its range.

Real types similarly have ranges (sets of representable values) associated
with them, but these ranges are implementation dependent. We do insist that
widening conversions from integral types can be made without loss of
information. For example every integer in Int16 must be exactly representable
in type Real32, because there is a widening conversion from Int16 to Real32.
These ranges have a minimum and maximum value. We say that a type has a value
that approximates the real value $x$ if $x$ is between the minimum and maximum
values for the type. Arithmetic operations on reals are defined as long as the
ncw type has a value that approximates the mathematical value.

The `as' construct is evaluated as follows.

\begin{itemize}
\item Conversions from integral types to integral types is defined only if the
value of the operand can be represented in the new type.

\item Conversions from real to real types are defined only if the value of the
operand can be approximated by a value in the new type.

\item Conversions from real types to integral types involve rounding the value
of the operand down to the closest integer that is less than or equal to the
value of the operand. The conversion is defined only if that integer is
representable in the new type.
\end{itemize}

\section{Genericity}

Classes and interfaces can be parameterized by \textquotedblleft generic
parameters\textquotedblright. The effect is a little like that of Java's
generic classes or C++'s template classes. Classes and interfaces may be
parameterized, in general, by other classes and interfaces, values of
primitive types, for example integers, and objects.

Programs using generics can be expanded to programs that do not use generics
at all. For example a program

\begin{code}
(class K ... class)

(class G\{ type T\ \} ...T... class)

obj g :\ G\{K\} := ...
\end{code}

Expands to

\begin{code}
(class K\ ... class)

obj k :\ K

(class G0 ...T... class)

obj g :\ G0 := ...
\end{code}

Generic parameters may be one of the following

\begin{itemize}
\item Nongeneric Types

\item Nongeneric Classes
\end{itemize}

%

\begin{align*}
\mathit{GParams}  &  \rightarrow\left\{  \;\mathit{GParam}^{+,}\;\right\} \\
\mathit{GParam}  &  \rightarrow\mathbf{type}\;\mathit{Name}\;\underline{[}%
\mathbf{extends}\;\mathit{Type}\underline{]}%
\end{align*}
%

\[
\mathit{GArgs}\rightarrow\{\;\mathit{Type}^{+,}\;\}
\]


\section{Examples}

\begin{code}
(\textbf{class} FIFO \{\textbf{type} T \textbf{extends} primitive\}
(\textbf{in} capacity :\ int16)

\begin{indent}
\item ~

\item \textbf{public} \textbf{proc} deposit(\textbf{in} value : T)

\item \textbf{public} \textbf{proc} fetch(\textbf{out} value : T)

\item \ 

\item \textbf{private} \textbf{obj} a :\ T(capacity)

\item \textbf{private} \textbf{obj} front := 0

\item \textbf{private} \textbf{obj} size := 0

\item \ 

\item (\textbf{thread}

\begin{indent}
\item (\textbf{wh} \textbf{true}

\begin{indent}
\item (\textbf{accept}

\item \qquad deposit(\ \textbf{in} value :\ T\ ) \textbf{when} size\
%TCIMACRO{\TEXTsymbol{<}}%
%BeginExpansion
$<$%
%EndExpansion
\ capacity

\item \qquad\qquad a[\ (front +\ size]\ \%\ capacity )\ := value

\item \qquad\qquad size := size + 1

\item
%TCIMACRO{\TEXTsymbol{\vert}}%
%BeginExpansion
$\vert$%
%EndExpansion


\item \qquad fetch(\ \textbf{out} value :\ T\ )\ \textbf{when} size
%TCIMACRO{\TEXTsymbol{>}}%
%BeginExpansion
$>$%
%EndExpansion
\ 0

\item \qquad\qquad value := a[front]

\item \qquad\qquad front := (front + 1) \% capacity

\item \qquad\qquad size := size - 1

\item \textbf{accept})
\end{indent}

\item \textbf{wh})
\end{indent}

\item \textbf{thread})
\end{indent}

\textbf{class})
\end{code}

\section{Lexical issues\label{sec:LexicalIssues}}

Comments are either single-line, C++\ style

\begin{code}
//This is a comment
\end{code}

\noindent or nesting multiline Pascal style

\begin{code}
(* this (* is also

a *) comment\ *)
\end{code}

\noindent The former can be ended by the end of the file. The latter can not.

Some operators have unicode or keyword equivalents.%
\[%
\begin{tabular}
[c]{llll}%
\textbf{Ascii} & \textbf{Keyword} & \textbf{Unicode glyph} & \textbf{Unicode
code point (hexadecimal)}\\
\texttt{/%
%TCIMACRO{\TEXTsymbol{\backslash}}%
%BeginExpansion
$\backslash$%
%EndExpansion
} & \texttt{and} & $\wedge$ & 2227\\
\texttt{%
%TCIMACRO{\TEXTsymbol{\backslash}}%
%BeginExpansion
$\backslash$%
%EndExpansion
/} & \texttt{or} & $\vee$ & 2228\\
\texttt{\symbol{126}} & \texttt{not} & $\lnot$ & 00ac\\
\texttt{\symbol{126}=} & \texttt{not=} & $\neq$ & 2260\\
\texttt{\_%
%TCIMACRO{\TEXTsymbol{<}}%
%BeginExpansion
$<$%
%EndExpansion
} &  & $\leq$ or $\leqslant$ & 2264 or 2a7d\\
\texttt{%
%TCIMACRO{\TEXTsymbol{>}}%
%BeginExpansion
$>$%
%EndExpansion
\_} &  & $\geq$ or $\geqslant$ & 2265 or 2a7e\\
\texttt{=%
%TCIMACRO{\TEXTsymbol{>}}%
%BeginExpansion
$>$%
%EndExpansion
} &  & $\Rightarrow$ & 21d2\\
\texttt{%
%TCIMACRO{\TEXTsymbol{<}}%
%BeginExpansion
$<$%
%EndExpansion
=} &  & $\Leftarrow$ & 21d0\\
\texttt{%
%TCIMACRO{\TEXTsymbol{<}}%
%BeginExpansion
$<$%
%EndExpansion
=%
%TCIMACRO{\TEXTsymbol{>}}%
%BeginExpansion
$>$%
%EndExpansion
} &  & $\Leftrightarrow$ & 21d4\\
\texttt{'} &  & $'$ & 0027, 02B9, 02BC, 2019, 2032, or FF07\end{tabular}
\ \
\]

\end{document}
