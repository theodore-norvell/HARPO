%
\documentclass{article}%
\usepackage{amsmath}
\usepackage{amsfonts}
\usepackage{amssymb}
\usepackage{graphicx}%
\setcounter{MaxMatrixCols}{30}
%TCIDATA{OutputFilter=latex2.dll}
%TCIDATA{Version=5.50.0.2960}
%TCIDATA{CSTFile=Literate Programming Article.cst}
%TCIDATA{Created=Monday, March 27, 2006 10:55:29}
%TCIDATA{LastRevised=Tuesday, February 04, 2014 16:36:35}
%TCIDATA{<META NAME="GraphicsSave" CONTENT="32">}
%TCIDATA{<META NAME="SaveForMode" CONTENT="3">}
%TCIDATA{BibliographyScheme=Manual}
%TCIDATA{<META NAME="DocumentShell" CONTENT="Articles\Literate Programming Article">}
%TCIDATA{Language=American English}
%BeginMSIPreambleData
\providecommand{\U}[1]{\protect\rule{.1in}{.1in}}
%EndMSIPreambleData
\newtheorem{theorem}{Theorem}
\newtheorem{acknowledgement}[theorem]{Acknowledgement}
\newtheorem{algorithm}[theorem]{Algorithm}
\newtheorem{axiom}[theorem]{Axiom}
\newtheorem{case}[theorem]{Case}
\newtheorem{claim}[theorem]{Claim}
\newtheorem{conclusion}[theorem]{Conclusion}
\newtheorem{condition}[theorem]{Condition}
\newtheorem{conjecture}[theorem]{Conjecture}
\newtheorem{corollary}[theorem]{Corollary}
\newtheorem{criterion}[theorem]{Criterion}
\newtheorem{definition}[theorem]{Definition}
\newtheorem{example}[theorem]{Example}
\newtheorem{exercise}[theorem]{Exercise}
\newtheorem{lemma}[theorem]{Lemma}
\newtheorem{notation}[theorem]{Notation}
\newtheorem{problem}[theorem]{Problem}
\newtheorem{proposition}[theorem]{Proposition}
\newtheorem{remark}[theorem]{Remark}
\newtheorem{solution}[theorem]{Solution}
\newtheorem{summary}[theorem]{Summary}
\newenvironment{proof}[1][Proof]{\textbf{#1.} }{\ \rule{0.5em}{0.5em}}
%{
\makeatletter
\typeout{litprog.sty Copyright T.S.Norvell 1997-1998.}

%%%%%%%%%%%%%%%%%%%%%%%%
% The code environment %
%%%%%%%%%%%%%%%%%%%%%%%%
\def\code{%
\bgroup%
\parsep 0pt%
\parskip 0pt%
\itemsep 0pt%
\topsep 0pt%
\parindent 0pt%
\def\descriptionlabel##1{\hspace\labelsep \bf ##1}%
\def\description{\list{}{\parsep 0pt \parskip 0pt \itemsep 0pt \topsep 0pt%
       \parindent 0pt \labelwidth\z@ \itemindent\z@ \leftmargin 2em%
       \let\makelabel\descriptionlabel}}%
\let\enddescription\endlist%
	\par\noindent\hrulefill\par\sf}%
\def\endcode{\par\noindent\hrulefill\par\egroup}

\def\hyperref#1#2#3#4{{\em #1 (\S\ref{#4})}}

\def\lpfile#1\lpendfile{{\bf File: \tt #1}}
\def\lpexpand#1\lpendexpand{{\bf $\langle$\rm(\S#1)$\rangle$}}
\def\lpaddsto#1\lpendaddsto{{\bf Adds to: \rm (\S#1)}}
\def\lpdefines#1\lpenddefines{{\bf defines: \sf #1}}
\def\lpreplace#1\lpwith#2\lpendreplace{{\bf Replace:} #1 {\bf with} #2}

\def\lptab{$\rightarrow $\quad\quad}

%%%%%%%%%%%%%%%%%%%%%%%%%%
% The indent environment %
%%%%%%%%%%%%%%%%%%%%%%%%%%

\def\indent{\description}
\def\endindent{\enddescription}

%%%%%%%%%%%%%%%%%%%%%%%
% Framed environement %
%%%%%%%%%%%%%%%%%%%%%%%
\newlength{\frameWidth}
\newlength{\frameTopExtra}
\newlength{\frameBottomExtra}
\newsavebox{\chuck}

\newenvironment{framed}
{\setlength{\frameWidth}{\linewidth}%
 \addtolength{\frameWidth}{-2\fboxsep}%
 \addtolength{\frameWidth}{-2\fboxrule}%
 \setlength{\frameTopExtra}{1.0ex}%
 \setlength{\frameBottomExtra}{0ex}%
 \setlength{\frameTopExtra}{\baselinestretch \frameTopExtra}%
 \global\setbox\chuck\vbox\bgroup\hsize\frameWidth \@parboxrestore}%
{\par\vskip\z@\egroup\shortstack{{}\\[\frameTopExtra]\fbox{\usebox\chuck}%
\\[\frameBottomExtra]{}}}

\makeatother
%}

%BeginMSIPreambleData
\ifx\pdfoutput\relax\let\pdfoutput=\undefined\fi
\newcount\msipdfoutput
\ifx\pdfoutput\undefined\else
\ifcase\pdfoutput\else
\msipdfoutput=1
\ifx\paperwidth\undefined\else
\ifdim\paperheight=0pt\relax\else\pdfpageheight\paperheight\fi
\ifdim\paperwidth=0pt\relax\else\pdfpagewidth\paperwidth\fi
\fi\fi\fi
%EndMSIPreambleData
\begin{document}

\title{Annotations for verification of HARPOL [Draft version 0]}
\author{Theodore S\ Norvell\\Electrical and Computer Engineering\\Memorial University}
\maketitle

\begin{abstract}
This document indicates extensions to the basic HARPOL language that support verification.

\end{abstract}

%

\[%
\begin{tabular}
[c]{ll}%
$N\rightarrow E$ & An $N$ can be $E$\\
$\underline{(}E\underline{)}$ & Grouping\\
$E\;F$ & An $E$ followed by an $F$\\
$E^{\ast}$ & Zero or more $E$s\\
$E^{\ast F}$ & Zero or more $E$s separated by $F$s\\
$E^{+}$ & One or more $E$s\\
$E^{+F}$ & One or more $E$s separated by $F$s\\
$E^{?}$ & Zero or one $E$s\\
$\underline{\lbrack}E\underline{]}$ & Zero or one $E$s\\
$E\mid F$ & Either an $E$ or a $F$%
\end{tabular}
\ \
\]


\section{Classes and Objects}

\subsection{Programs}

A program is a set of classes, interfaces, and objects.%
\[
Program\rightarrow\underline{(}ClassDecl\mid IntfDecl\mid ObjectDecl\mid
ConstDecl\mid;\underline{)}^{\ast}%
\]


\subsection{Objects}

Fields may be ghost fields. Ghost fields may be used in specification
annotations and ghost code. Initialization expressions for ghost variables may
contain ghost variables. Likewise assignments to ghost variables and arguments
for ghost parameters may contain ghost variables.%
\[
ObjectDecl\rightarrow\underline{[}\mathbf{ghost}\underline{]}\;(\mathbf{const}%
\mid\mathbf{obj})\;Name\;\underline{[}:Type\underline{]}:=InitExp
\]


There is a new primitive type \textrm{Perm} which is only available for ghost
fields. Values of type float can be widened to Perm. Expressions of type Perm
are used to express degrees of ownership with $0$ meaning no ownership, $1$
meaning total ownership and any number strictly between $0$ and $1$ meaning
partial ownership (enough to read, but not enough to write). [Note:\ In the
future permission values may be extended to include infinitessimals of various
degrees, e.g. $0+4\varepsilon+10\varepsilon^{2}$. But for now, we will just
use simple fractions.]

\subsection{Classes and interfaces}

Constructor parameters may be ghost%
\[
CParam\rightarrow\underline{[}\mathbf{ghost}\underline{]}\;\mathbf{obj}%
\;Name:Type\mid\underline{[}\mathbf{ghost}\underline{]}\;\mathbf{in}%
\;Name:Type
\]


\subsection{Class Members}

Class members include invariants%
\begin{align*}
ClassMember &  \rightarrow Invariant\mid Claim\mid\cdots\\
Invariant &  \rightarrow\mathbf{invariant}\;Condition
\end{align*}
and claims. The claim indicates which locations belong to the object
initially. The invariant must hold initially (i.e. in the initial state of the
object) and at the end of each \textbf{with} command.

Method declarations may include pre- and postconditions as well as takes and
gives clauses. Parameters may be ghost.%
\begin{align*}
Method  &  \rightarrow Access\;\mathbf{proc}\;Name(\underline{(}%
\underline{[}\mathbf{ghost}\underline{]}\;Direction\;Name:\;Type\underline{)}%
^{\ast,})\underline{]}\underline{(}MethodSpecClause\underline{)}^{\ast}\\
MethodSpecClause  &  \rightarrow\mathbf{pre}\;Condition\\
&  \mid\mathbf{post}\;Condition\\
&  \mid\mathbf{takes}\;PermMap\\
&  \mid\mathbf{gives}\;PermMap\\
&  \mid\mathbf{borrows}\;PermMap
\end{align*}


\subsection{Conditions}

Conditions are boolean expressions that may refer to ghost variables. Within a
postcondition, expressions may be marked with a postfix apostrophe to refer to their
final values. E.g.\ \texttt{x} is an initial value while \texttt{x' }is a
final value.
 The precedence of the apostrophe operator is just below unary -. Thus \texttt{-x'} is parsed as \texttt{-(x')}.
\subsection{Permission maps}

A permission map describes a mapping from locations to permission amounts.
\[
PermMap\rightarrow LocSet\mid LocSet@Expression\mid PermMap,PermMap
\]
The default permission is 1, i.e. full ownership; i.e., if a permission map
consists only of a location set, then the each location in the set is mapped
to 1.\ The thrid possibility indicates the sum of two permission maps. E.g.
$a@0.5,a@0.5$ is equivalent to $a@1.0$.

The location set may just be one location as specified by a single object id,
or it can be a set of locations.%
\[
LocSet\rightarrow ObjectId\mid\{Name:Set\;\underline{[}Guard\underline{]}%
\;\mathbf{do}\;ObjectId\}
\]
The name here is bound within the location set expression. The range of the
name is given by a set expression, which can be any of the following. The
object id's must be of premitive type.%
\[
Set\rightarrow\left\{  Expression,..Expression\right\}  \mid\left\{
Expression,..,Expression\right\}  \mid Expression
\]
The first form is exclusive of its upper bound. The second form is inclusive.
The third form is equivalent to $\left\{  0,..Expression\right\}  $

Within location sets and permission maps, expressions may include ghost
variables.%
\[
ObjectId\rightarrow Name\mid ObjectId[Expression]\mid ObjectId.Name
\]


\section{Threads}

Threads may start with a claim
\begin{align*}
Thread &  \rightarrow(\mathbf{thread}\;\underline{[}Claim\underline{]}%
\;Block\;\underline{[}\mathbf{thread}\underline{]})\\
Claim &  \rightarrow\mathbf{claim}\;PermMap
\end{align*}
The locations of an object (i.e.\ its primitive fields and the primitive items
of any array fields) may be claimed by a thread. The sum of the claims for the
various threads and the initial claim of the class must not be more than 1 for
any location.

\subsection{Statements and Blocks}

\begin{itemize}
\item Statements include assert and assume%
\begin{align*}
Command  &  \rightarrow\mathbf{assert}\;Condition\\
Command  &  \rightarrow\mathbf{assume}\;Condition
\end{align*}


\item Method call statements may have ghost arguments, but there is no special
marking of them%
\begin{align*}
Command  &  \rightarrow ObjectId.Name(Args)\\
&  \mid Name(Args)\\
Args  &  \rightarrow\underline{[}Expressions\underline{]}%
\end{align*}


\item Loops may have loop invariants%
\begin{align*}
Command  &  \rightarrow\left(  \mathbf{while}~Expression\;\underline{\{}%
Invariant\underline{\}}\;\underline{[}\mathbf{do}\underline{]}%
\;Block\;\underline{[}\mathbf{while}\underline{]}\right) \\
&  \mid\left(  \mathbf{for}\;Name:Bounds\;\underline{[}Invariant\underline{]}%
\;\underline{[}\mathbf{do}\underline{]}\;Block\;\underline{[}\mathbf{for}%
\underline{]}\right)
\end{align*}


\item Parallel blocks may begin with claims%
\begin{align*}
Command &  \rightarrow\left(  \mathbf{co}\;\underline{[}Claim\underline{]}%
\ Block\;\underline{(}||\;\underline{[}Claim\underline{]}\;Block\underline{)}%
^{+}\;\underline{[}\mathbf{co}\underline{]}\right)  \\
&  \mid\left(  \mathbf{co}\;Name:Bounds\;\underline{[}\mathbf{do}%
\underline{]}\;\underline{[}Claim\underline{]}\;Block\;\underline{[}%
\mathbf{co}\underline{]}\right)
\end{align*}
At the start of a \textbf{co} command the thread's permission map is split
according to the claims. Thus the sum of the claims must not exceed the
initial permissions of the parent thread; unclaimed permission remains with
the parent thread. At the end of a \textbf{co}, the permissions of the child
threads are added back to the permissions of the parent thread.

\item Method implementation (note that the vertical bar here is a terminal)%
\begin{align*}
Command  &  \rightarrow\left(  \mathbf{accept}\;MethodImp\;\underline{(}%
|\;MethodImp\underline{)}^{\ast}\;\underline{[}\mathbf{accept}\underline{]}%
\right) \\
MethodImp  &  \rightarrow Name(\;\underline{(}%
Direction\;Name:Type\underline{)}^{\ast,}\;)\;\underline{[}Guard\underline{]}%
\;Block_{0}\;\underline{[}\mathbf{then\;}Block_{1}\underline{]}\\
Guard  &  \rightarrow\mathbf{when}\;Expression
\end{align*}


\item With commands have take and give%
\[
Command\rightarrow\left(  \mathbf{with}\;Exp\;\underline{[}\mathbf{takes}%
\;PermMap\underline{]}\;\underline{[}Guard\underline{]}\;\underline{[}%
\mathbf{do}\underline{]}\;Block\;\underline{[}\mathbf{gives}%
\;PermMap\underline{]}\;\underline{[}\mathbf{with}\underline{]}\right)
\]
Within a \textbf{with} command, the permissions of the tread are augmented by
the permissions held by the lock object.\ E.g. $x:=2$ is allowed if the
permission of the lock object on $x$ and the permission of the thread on $x$
sum to $1$.
\end{itemize}

\section{Conditions}

Conditions are just ordinary boolean expressions with the addition of the
following operations

\begin{itemize}
\item canRead(LocSet) :\ a boolean; true if every location in the LocSet can
be read (permission $>0.0$).

\item canWrite(LocSet) :\ a boolean; true if every location in the LocSet can
be written (permission is 1.0).

\item permission(ObjectId)\ :\ a number from $0.0$ to $1.0$.
\end{itemize}


\end{document}
